\chapter{Utilerías JS}

\section{zend\_django.js}

\subsection{Date}

\begin{description}
	\item[String asMySQL()] Devuelve la fecha en formato MySQL YYYY-MM-DD
	\item[String asMx()] Devuelve la fecha en formato europeo DD-MM-YYYY
	\item[String theTime()] Devuelve la hora en formato hh:mm
	\item[Date fromMX(String date)] Recibe una cadena en formato DD-MM-YYYY y devuelve el objeto tipo Date correspondiente
	\item[Date addDays(int dias)] Suma un numero de días a la fecha
\end{description}

\subsection{Number}

\begin{description}
	\item[String asMoney()] Devuelve el número como cadena en formato de moneda (con dos decimales)
\end{description}

\subsection{clsApp : App}

\begin{description}
	\item[void checkInputIn(String idcontainer)] Activa todas las casillas de verificación contenidas en el contenedor con id = idcontainer
	\item[void uncheckInputIn(String idcontainer)] Desactiva todas las casillas de verificación contenidas en el contenedor con id = idcontainer
	\item[void openPanel(String body, String title, boolean close=true, String footer=null,]
	\item[\quad String idmodal=``modal-panel-message'')] Genera un nuevo panel bootstrap con el contenido de body y el identificador indicado en idmodal
	\item[void closePanel(String idmodal=``modal-panel-message'')] Cierra el panel bootstrap identificado por idmodal
	\item[void setUIControls] Genera los controles UI para navegadores que no soportan controles HTML5 para controles input tipo date
	\item[void setReadOnlyForm(String container\_selector=``\#main-form'')] Aplica readonly a los controles de formulario contenidos en el contenedor indicado por container\_selector
	\item [void showPrivacyPolicy()] Abre en un panel bootstrap el contenido de politica de privacidad, establecido en el elemento HTML con el selector \textit{\#privacy-policy-template}
	\item[void showDeletingConfirmation(String url, String elemento=``elemento'', String pre\_elemento=``el'')] Muestra el panel bootstrap de confirmación de eliminación de `el` `elemento`, el boton de \textit{Aceptar} es un hipervínculo a url
	\item[boolean isEmpty(variant valor)] Indica si es una cadena vacia o vale 0 (cero)
	\item[boolean validate\_required\_fields(String container)] Verifica si todos los controloes dentro del container (selector) tienen un valor establecido
\end{description}